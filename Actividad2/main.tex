\documentclass{article}
\usepackage[utf8]{inputenc}

\title{Actividad 2 Movimiento de proyectiles}
\author{ Glenda Carranco}
\date{ 30 August 2019}

\begin{document}

\maketitle

\section{Movimiento de proyectiles}

Movimiento parabolico:

Se le llama movimiento parabólico al movimiento realizado por un objeto cuya trayectoria se asemeja a una parábola. Se corresponde con la trayectoria ideal de un proyectil que se mueve en un medio que no ofrece resistencia al avance.

Caracteristicas del tiro parabolico:

\begin{itemize}
    \item Conociendo la velocidad inicial, el angulo de inclinacion inical y la diferencia del alturas, se conocera toda la trayectoria.
    \item Lon ángulos de salida y llegada son iguales
    \item El mayor alcance se logra con ángulo de salida de 45
\end{itemize}
 
 
 
 \section{Ecuaciones del Movimiento parabolico.}
 Hay 2 Ecuaciones que rigen el movimiento parabólico:
\begin{equation}
          v=V_0 cos\theta i + v_o sin\theta j
         \end{equation}
          
\begin{equation}
              a=-gj
\end{equation}

Ecuación de la velocidad:
\begin{equation}
a = \frac{dv}{dt} = -gj
    \end{equation}
    \begin{equation}
        v_0 = v_0_x i + v_o_y j
    \end{equation}

Ecuación de la Posición:
\begin{equation}
    v = \frac{dr}{dt} = v_0_x i + (v_0_y - gt)j
\end{equation}

\begin{equation}
    r(0) = x_0i + y_oj
\end{equation}

\section{Movimiento Parabólico de rozamiento}
Cuando se considera el rozamiento la trayectoria es casi una parábola. El estudio de la trayectoria en ese caso es considerado por la balística.

\section{Actividad en clase}

En la actividad en clase, usamos un codigo para calcular las diferentes situaciones con diferentes datos.

(1) Mostrar una tabla con entradas y salidas para demostrar que al lanzar un objeto con un angulo de 45 grados, tendria el alcance maximo (x)

\begin{table}[]
\begin{tabular}{|c|c|c|c|c|c|c|}
\hline
\textbf{Ángulo} & \textbf{Velocidad I} & \textbf{Posición X} & \textbf{Posición y} & \textbf{Velocidad} & \textbf{Theta} & \textbf{Tiempo} \\ \hline
35              & 15 m/s               & 21.5745 m           & 0.00 m              & 15.00 m/s          & -35.00         & 1.75 s          \\ \hline
40              & 15 m/s               & 22.6103 m           & 0.00 m              & 15.00 m/s          & -40.00         & 1.9677 s        \\ \hline
45              & 15 m/s               & 22.9591 m           & 0.00 m              & 15.00 m/s          & -45.00         & 2.1646 s        \\ \hline
50              & 15 m/s               & 22.6103 m           & 0.00 m              & 15.00 m/s          & -49.50         & 2.3450 s        \\ \hline
55              & 15 m/s               & 21.5745 m           & 0.00 m              & 15.00 m/s          & -55.00         & 2.5076 s        \\ \hline
\end{tabular}
\end{table}

\section{Bibliografia}

\begin{itemize}
    \item https://www.tablesgenerator.com/#
    \item https://en.wikipedia.org/wiki/Projectile_motion
    \item http://www.proyectosalonhogar.com/Enciclopedia_Ilustrada/Ciencias/Movimiento_Proyectiles.htm
    
\end{itemize} 













\end{document}
